% LaTeX file for resume

\documentclass[line,margin]{cv}

\usepackage[usenames,dvipsnames]{color}
% margin control
\usepackage{scrextend}

% hyperlinks
\usepackage{hyperref}
\hypersetup{
  colorlinks=true,
  urlcolor=Black,
  urlbordercolor={0 0 0}
}

% line spacing
\usepackage{setspace}
\linespread{1.3}

% for compact itemization
\usepackage{enumitem}
\setlist{nolistsep}
\setlength{\itemsep}{-.08ex}


%%%%%%%%%%%%%%%%%%%%%%%%%%%%%%%%%%%%%%%%%%%%%%%%%%%%%%%%%%%%%
%%%%%%% My settings
%%%%%%%%%%%%%%%%%%%%%%%%%%%%%%%%%%%%%%%%%%%%%%%%%%%%%%%%%%%%%

% page margin
\addtolength{\textwidth }{50pt}
\addtolength{\oddsidemargin }{-30pt}

% set paragraph margin (spacing)
\setlength{\parskip}{1.5ex plus 0.3ex minus 0.3ex}

% linespread
\linespread{1.1}

% Indented block
\newenvironment{block}
{
  \begin{addmargin}[2em]{0em}% 2em left, 0em right
}
{
  \end{addmargin}
}

% Section font becomes Small Capitals
\renewcommand{\sectionfont}{\scshape}

\newcommand{\myemph}{\textbf}

% Use a smaller bullet for listing
\renewcommand{\labelitemi}{$\vcenter{\hbox{\tiny$\bullet$}}$}

% links
\newcommand{\SUNY}{\href{http://www.stonybrook.edu/}{State University of New York at Stony Brook}}
\newcommand{\Google}{\href{https://www.google.com/}{Google}}
\newcommand{\GoogleBrain}{\href{http://research.google.com/teams/brain/}{Google Brain}}
\newcommand{\GoogleDeepMind}{\href{https://www.deepmind.com/}{Google DeepMind}}
\newcommand{\Facebook}{\href{https://www.facebook.com/}{Facebook}}
\newcommand{\SUNYSchool}{\href{www.cs.stonybrook.edu}{Computer Science Department}}
\newcommand{\Fudan}{\href{http://www.fudan.edu.cn/englishnew/}{Fudan University}}
\newcommand{\FudanSchool}{\href{http://www.cs.fudan.edu.cn/}{School of Computer Science}}
\newcommand{\MyMail}{\href{mailto:alan.yt.tian@gmail.com}{alan.yt.tian@gmail.com}}
\newcommand{\MSRA}{\href{http://research.microsoft.com/en-us/labs/asia/default.aspx}{Microsoft Research Asia}}
\newcommand{\DMAS}{\href{http://research.microsoft.com/en-us/groups/dmas/}{Data Management, Analytics and Services Group}}
\newcommand{\Haixun}{\href{http://haixun.olidu.com/}{Dr.\ Haixun Wang}}
\newcommand{\WSM}{\href{http://research.microsoft.com/en-us/groups/wsm/}{Web Search and Mining Group}}
\newcommand{\Zhongyuan}{\href{http://www.wangzhongyuan.com/}{Dr.\ Zhongyuan Wang}}
\newcommand{\Xiaomeng}{\href{https://www.linkedin.com/in/xiaomengban/}{Dr.\ Xiaomeng Ban}}
\newcommand{\Skiena}{\href{https://www3.cs.stonybrook.edu/~skiena/}{Prof.\ Steven Skiena}}
\newcommand{\GDM}{\href{http://gdm.fudan.edu.cn}{Graph Data Management Lab}}
\newcommand{\Yanghua}{\href{http://gdm.fudan.edu.cn/GDMWiki/Wiki.jsp?page=Yanghuaxiao}{Prof.\ Yanghua Xiao}}
\newcommand{\sgouws}{\href{http://research.google.com/pubs/StephanGouws.html}{Dr.\ Stephan Gouws}}
\newcommand{\jesseengel}{\href{https://ai.google/research/people/JesseEngel}{Dr.\ Jesse Engel}}
\newcommand{\ICPC}{\href{http://https://icpc.global//}{ACM-ICPC}}
\newcommand{\Codejam}{\href{http://code.google.com/codejam/}{Google Code Jam}}


% compact itemize
\newenvironment{citemize}
{ \begin{itemize}[itemsep=-0.4ex] }
{ \end{itemize} }


%%%%%%%%%%%%%%%%%%%%%%%%%%%%%%%%%%%%%%%%%%%%%%%%%%%%%%%%%%%%%
%%%%%%% Real staffs
%%%%%%%%%%%%%%%%%%%%%%%%%%%%%%%%%%%%%%%%%%%%%%%%%%%%%%%%%%%%%

\begin{document}

\name{Yingtao Tian}

\address{\textit{E-mail:} \MyMail}

\begin{resume}

\section{Experience}

  {\bf \GoogleDeepMind{} (formerly \GoogleBrain)} {\itshape Research Scientist} \hfill 2019 - Now
  \begin{block}
    Research on generative models and more (see projects and publications)
  \end{block}

  {\bf Internships} \hfill Periodically 2013 - 2018
  \begin{block}
    \GoogleBrain{},  \Facebook{}, \Google{}, \MSRA{}.
  \end{block}

\section{Education}

  {\bf \SUNY}, New York, U.S. \hfill 2014 - 2019
  \begin{block}
    Ph.D, Computer Science. Advisor: \Skiena
  \end{block}

  {\bf \Fudan}, Shanghai, China. \hfill 2010 - 2014

  \begin{block}
    B.Sc., Computer Science and Technology 
  \end{block}



\section{Major Publications}

  In my research I aim at combining \textbf{computational approach}, with {generative/creative settings} with artists, culture, humanities and designer's consideration, in a way that the considerations are met while machine learning can help boosting the performances. To do so, I propose following techniques and tools that have both {addressed the needs in creative settings} and \emph{advances in core machine learning}. See more details in my research statement {\url{https://alantian.net/research_statement.pdf}}.

  \textbf{(1) Generating Artifacts with Artistical Discretion}:  Co-design of generating algorithm and generating algorithm is needed to exercise such discretion.
  In my work \underline{ES-CLIP} and \underline{ES-3D} I {design one particular abstract art} form, framing as synthesizing painting by placing semi-transparent triangles respectively on 2D canvas and 3D space. In my work \underline{Simultaneous Multiple-Prompt} I designed a fully end-to-end differential approach that can make any text-to-image models adapting to {multiple text prompts coherently}. In our work \underline{Evolving Collective AI} we bridge {biological collective intelligence} with artificial intelligence, by constructing {designs of agents} inspired by the biological ants

  \underline{ES-CLIP}: {Modern Evolution Strategies for Creativity: Fitting Concrete Images and Abstract Concepts}.
  \emph{Yingtao Tian}, David Ha.
  \emph{In the Proceedings of the The 11th International Conference on Artificial Intelligence in Music, Sound, Art and Design (EvoMUSART) 2022.}

  \underline{ES-3D}: {Evolving Three Dimension (3D) Abstract Art: Fitting Concepts by Language}.
  \emph{Yingtao Tian} \emph{Working paper}

  \underline{Simultaneous Multiple-Prompt}: {Simultaneous Multiple-Prompt Guided Generation Using Differentiable Optimal Transport}.
  \emph{Yingtao Tian}, David Ha, Marco Cuturi. \emph{In the Proceedings of the twelfth International Conference on Computational Creativity, ICCC'22}

  \underline{Evolving Collective AI}: {Evolving Collective AI: Simulation of Ants Communicating via Chemicals}.
  Ryosuke Takata, Yujin Tang, \emph{Yingtao Tian}, Norihiro Maruyama, Hiroki Kojima, Takashi Ikegami. \emph{The 2023 Conference on Artificial Life}

  \textbf{(2) Machine learning-Boosted Tools for Historical and Cultural works}:   We encounter a lot of works related to historical and cultural aspects where faithful adherence to a few paradigms is a must.
  My work \underline{KaoKore} and \underline{ARC Ukiyo-e Face} organize {facial expression images} from 14-16 century and 17-19 century Japanese artworks and enable {computational analysis} to such artworks. Our work \underline{MingOfficial} proposes LLMs and GNNs to learn representations from structured data and raw historical record, enabling {identifing historical figures} with interesting traits.
  We also, in our work \underline{Digital Typhoon}, organize the longest consecutive typhoon satellite image dataset in modern history to helpf {weather prediction}. 
  Forthermore, by designing proper diffusion models, I propose \underline{DiffCJK} to {generate glyphs of new calligraphy and typology styles} for hundreds of thousands CJK characters.

  \underline{KaoKore}: {KaoKore: A Pre-modern Japanese Art Facial Expression Dataset}.
  \emph{Yingtao Tian}, Chikahiko Suzuki, Tarin Clanuwat, Mikel Bober-Irizar, Alex Lamb, Asanobu Kitamoto. \emph{In the Proceedings of the Eleventh International Conference on Computational Creativity, ICCC'20}

  \underline{ARC Ukiyo-e Face}: {Ukiyo-e Analysis and Creativity with Attribute and Geometry Annotation}.
  \emph{Yingtao Tian}, Tarin Clanuwat, Chikahiko Suzuki, Asanobu Kitamoto. \emph{In the Proceedings of the Eleventh International Conference on Computational Creativity, ICCC'21}

  \underline{MingOfficial}: {MingOfficial: A Ming Official Career Dataset and a Historical Context-Aware Representation Learning Framework}
  You-Jun Chen, Hsin-Yi Hsieh, Yu Tung Lin, \emph{Yingtao Tian}, Bert Chan, Yu-Sin Liu, Yi-Hsuan Lin, Richard Tzong-Han Tsai. \emph{The 2023 Conference on Empirical Methods in Natural Language Processing (EMNLP 2023)}

  \underline{Digital Typhoon}: {Digital Typhoon: Long-term Satellite Image Dataset for the Spatio-Temporal Modeling of Tropical Cyclones}.
  Asanobu Kitamoto, Jared Hwang, Bastien Vuillod, Lucas Gautier, \emph{Yingtao Tian}, Tarin Clanuwat. \emph{Thirty-seventh Conference on Neural Information Processing (NeurIPS 2023) Systems Datasets and Benchmarks Track}

  \underline{DiffCJK}: {DiffCJK: Conditional Diffusion Model for High-Quality and Wide-coverage CJK Character Generation}.
  \emph{Yingtao Tian} \emph{International Conference on Computational Creativity (ICCC) 2024}

  \textbf{(3) Advancing Machine Learning Techniques}: The advances in the creative models cannot be without advances in core machine learning research.
  Many works concern {evolution strategy}, an gradient-free optimization technique other than the more commonly used gradient based methods. 
  We propose the \underline{EvoJAX}, a scalable, general purpose, hardware-accelerated {neuroevolution toolkit}; \underline{NeuroEvoBench}, a new {benchmark of evolutionary optimization methods} for Deep Learning and meta-learned tasks.  
  We further extend evolution strategy with LLM: \underline{EvoLLM} implements a type of {black-box recombination operator using LLM through Evolution}, and \underline{EvoTransformer} flexibly characterizes a family of {evolution strategies with Transformer architecture}.
  Some work also concerns {reinforcement learning}: In our work \underline{DEIR} we propose a theoretically-backend intrinsic reward from conditional mutual information to quickly learn policies.


  \underline{EvoJAX}: {EvoJAX: Hardware-Accelerated Neuroevolution}
  Yujin Tang, \emph{Yingtao Tian}, David Ha.   \emph{In the Proceedings of the Genetic and Evolutionary Computation Conference (GECCO) 2022}

  \underline{NeuroEvoBench} {NeuroEvoBench: Benchmarking Neuroevolution for Large-Scale Machine Learning Applications}.
  Robert Tjarko Lange, Yujin Tang, \emph{Yingtao Tian}. \emph{Thirty-seventh Conference on Neural Information Processing (NeurIPS 2023) Systems Datasets and Benchmarks Track}

  \underline{EvoLLM}: {Large Language Models As Evolution Strategies}.
  Robert Tjarko Lange, \emph{Yingtao Tian}, Yujin Tang. \emph{The Genetic and Evolutionary Computation Conference (GECCO) 2024}

  \underline{EvoTransformer}: {Evolution Transformer: In-Context Evolutionary Optimization}.
  Robert Tjarko Lange, \emph{Yingtao Tian}, Yujin Tang. \emph{The Genetic and Evolutionary Computation Conference (GECCO) 2024}

  \underline{DIER}: {DEIR: Efficient and Robust Exploration through Discriminative-Model-Based Episodic Intrinsic Rewards}.
  Shanchuan Wan, Yujin Tang, \emph{Yingtao Tian}, Tomoyuki Kaneko. \emph{In the Proceedings of the 32nd International Joint Conference on Artificial Intelligence, IJCAI-23}



\section{(Old) Awards in Competitive Programming}
  \myemph{27th place}, 35th Annual World Final of the \ICPC \,, 2011

  \myemph{Gold Medal}, \ICPC\ Asia Chengdu Regional Contest, 2011

  \myemph{Championship and Gold medal}, \ICPC\ Asia Amritapri Regional Contest, 2010

\end{resume}
\end{document}
